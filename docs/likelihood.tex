\subsection{Maximum Likelihood Estimators}
\label{sec:maxmean}

\subsubsection{Gaussian Data}

The joint likelihood for a set of Gaussian-distributed data is simply
the product of the individual probabilities:

\begin{eqnarray}
L \equiv \prod{p_k} &\propto& \prod_k{e^{-(x_k - \mu)^2/2\sigma^2_k}} \\
                    &=& e^{-\sum_k{(x_k - \mu)^2/2\sigma^2_k}} \\
                    &\equiv& e^{-\chi^2/2} \\
\end{eqnarray}

The maximum likelihood estimate of a set of parameters is therefore,
by definition, the parameters that minimize $\chi^2$.

\subsubsection{Poisson Data}

What if our data are Poisson distributed, say a an X-ray image, where
each pixel is the number of photons, and does not necessarily contain
enough photons that the gaussian approximation is even remotely
justified?  In this case, the joint likelihood is still the product of
the individual probabilities, but where each probability is given by
the Poisson pdf:

\begin{equation}
L \equiv \prod_k{p_k} = \prod_k{{{e^{-\mu_k}\mu_k^k}\over{k!}}}
\end{equation}

Here $\mu_k$ is the mean expected count rate in pixel $k$, the sum of
the model prediction and the background count rate.

\subsubsection{Joint Likelihood Calculation}

For chains that involve a combination of Gaussian and
Poisson-distributed data, the likelihood is straightforwardly calculated as


\begin{equation}
L \equiv \prod_i{L_i},
\end{equation}

where $i$ ranges over the independent datasets, with $L_i$ calculated
as above for Gaussian or Poisson datasets.  For purposes of reported
goodness of fit ($\chisq$), an equivalent $\chisqnor$ distributed is
constructed for Poisson data, noting that:

\begin{equation}
f_N(\chisqnor) = {1\over{2\Gamma(N/2)}}e^{-\chisqnor/2}\left({\chisqnor\over{2}}\right)^{N/2-1}
\end{equation}

and

\begin{equation}
P(\mu;k) = {1\over{k!}}e^{-\mu} \mu^{k}
\end{equation}

with the identification of

\begin{equation}
\chisqnor = 2\mu
\end{equation}

and 

\begin{equation}
N = 2(k + 1)
\end{equation}

