\section{Quickstart Guide}

\subsection{Introduction}

Currently, \climax\ consists of a single binary, \$\{CLIMAX\_SRC\}/bin/\climaxb\, which is
entirely driven by parameter files; no command-line options except the
parameter file name are supported, so that runs can be exactly
reproduced from the contents of the parameter file.  The binary can be run like:

\begin{myindentpar}{3cm}
climaxRunManager file=myParameterFile
\end{myindentpar}

Additionally, a simple Python hook is provided with the code, which
is compiled into the loadable module
\$\{CLIMAX\_SRC\}/python/climaxPyTest.so (if PYTHON\_INC\_PATH is
defined in Makefile.directives when the code is built).  To use the
code from Python, do:

\begin{myindentpar}{3cm}
import climaxPyTest\\
res = climaxPtTest.run('myParameterFile')
\end{myindentpar}

where 'res' will contain a dictionary of parameter values, 1-sigma
confidence limits, and other information.

One goal of the code documentation is that it be self-describing as
much as possible; this means that any error in the parameter file
should generate a usage message with information on valid options, or
instructions on how to fix syntactical errors.  Run \climaxb\ with
no options to get started.

\subsection{Theory of Operation}

The general structure of parameter files is designed around defining
datasets and models, using the commands {\tt adddataset} and {\tt
  addmodel}.  To see a list of recognized datasets (or models), just
enter one of these commands with in invalid type.  Datasets in
\climax\ are aware of which models can apply to them, and by default
all models will be fit to any datasets to which they apply.  This
behavior can be modified by using the {\tt exclude} command to exclude
models from particular datasets (see \S~\ref{sec:datasets}).

Once datasets and models are declared, \climaxb\ can do one of several
things, depending on control parameters in the parameter file: by
default it will run a Markov chain for any variable model components,
but it can also display datasets, models and residuals, compute
chi-squared for defined datasets and models, load and display output
chains, or generate simulated data for a dataset type if the dataset
supports it.  What it does is determined by global control parameters,
detailed in \S~\ref{sec:control}.

\subsection{Defining Datasets}
\label{sec:datasets}

Once a dataset is declared, the user can set individual parameters of
the dataset by name.  For example, to declare an interferometric UVF
dataset, you would use:

\begin{myindentpar}{3cm}
adddataset type=uvf name=duvf;
\end{myindentpar}

which declares a new dataset of type 'uvf' and associates the name
'duvf' with it. Parameters of this dataset can subsequently be
declared like:

\begin{myindentpar}{3cm}
duvf.file = /path/to/my/uvf/filename;
\end{myindentpar}

To see a list of recognized parameters for a dataset, just set an
invalid parameter for that dataset.

If desired, models can be excluded from particular datasets by using
the \code{exclude} meethod.  For example,

\begin{myindentpar}{3cm}
duvf.exclude(model1);
\end{myindentpar}

would prevent \code{model1} from being fit to \code{duvf}.  This
allows greater flexibility in the way that chains are run.  For
example, for data from an instrument like the SZA, you can
dramatically increase the speed of fitting cluster and point source
models by loading the data twice, once with a uvmax of $2 k\lambda$
(short-baseline data) and once with a uvmin of $2 k\lambda$
(long-baseline data), fitting a point-source model to both, but
excluding the cluster model from the long baseline data, which have no
sensitivity to it anyway.  This is because the cluster fitting will
only require Fourier inversion of the short-baseline data, which can
be coarsely gridded and quickly inverted, while the point-source
models are fit directly on the Fourier plane and require no inversion
of the models for either dataset.

\subsection{Defining Models}
\subsubsection{Adding Models}
Models can contain both parameters and potentially variable
components.  To see a list of recognized parameters and components for
a model, just set an invalid parameter for that model.

If a component is assigned a specific value in the
parameter file, it is held fixed at that value.  For example

\begin{myindentpar}{3cm}
addmodel type=betamodel name=model1;\\
model1.beta = 0.8;
\end{myindentpar}

defines a \betamodel\ with a fixed value for the $\beta$ parameter.
If a component is instead specified with a prior, then that component
is treated as variable.  For example,

\begin{myindentpar}{3cm}
model1.beta = 0.6:0.8;
\end{myindentpar}

would define a \betamodel\ with a uniform prior of $\{0.6, 0.8\}$ for
the $\beta$ parameter.  On a more subtle note, for the Metropolis-Hastings algorithm,
\climax\ by default selects the center of any uniform prior as the
starting mean for the jumping distribution.  You can override this
behavior by optionally specifying a starting mean explicitly.  The
specification

\begin{myindentpar}{3cm}
model1.beta = 0.65:0.6:0.8;
\end{myindentpar}

for example would set the same uniform prior but initialize the
jumping distribution with a mean of 0.65.  Alternately, you can assign
a Gaussian prior, via:

\begin{myindentpar}{3cm}
model1.beta = 0.7 +- 0.1;
\end{myindentpar}

Model components with units must be specified in a valid unit for that
component.  For example, to specify the radio normalization of the
\betamodel\ you could use:

\begin{myindentpar}{3cm}
model1.Sradio = -2000:-500 muK;
\end{myindentpar}

to set a prior in micro-Kelvin.  Enter an invalid unit to see what
units are supported for each model component.  Additionally,
\climax\ will allow any numerical scaling from a recognized unit.  For
example:

\begin{myindentpar}{3cm}
m.m500 = 1e15:5e15 Msolar;
\end{myindentpar}

and 

\begin{myindentpar}{3cm}
m.m500 = 1:5 1e15Msolar;
\end{myindentpar}

are both valid specifications for a model with mass parameter 'm500'.

\subsubsection{Removing Models}

Models can be removed from the data prior to any operations by using
the \code{remmodel} keyword.  Models are defined just as with
\code{addmodel}, but are subtracted from the data prior to fitting or
imaging.

\subsubsection{Correlating Model Parameters}

You can set any model parameter equal to another (symbolic) model
parameter, provided they are of the same type.  This links the two
parameters together, with only the first parameter sampled by the
Markov chain.  For example:

\begin{myindentpar}{3cm}
addmodel type=betamodel name=beta;\\
addmodel type=ptsrc name=pt;\\
\\
beta.xoff = -0.01:0.01 deg;\\
beta.yoff = -0.01:0.01 deg;\\
\\
pt.xoff = beta.xoff;\\
pt.yoff = beta.yoff;
\end{myindentpar}

would fit a cluster and a point source, with the location of the point
source fixed to the cluster center.  This can be useful in other
contexts too; for example, i n combination with model exclusion (see
\S\ref{sec:datasets}), it could be used to tie together physical model
parameters when two datasets have an unphysical offset between them,
e.g.:

\begin{myindentpar}{3cm}
adddataset type=uvf name=sz;\\
adddataset type=xrayimage name=xray;\\
\\
addmodel type=betamodel name=msz;\\
msz.Sradio = -2000:0 muK;\\
msz.xoff = -0.01:0.01 deg;\\
msz.yoff = -0.01:0.01 deg;\\
msz.beta = 0.1:0.8;\\
msz.thetaCore = 0.1:1.0';\\
\\
addmodel type=betamodel name=mx;\\
mx.Sxray   = 0:2000 counts;\\
mx.xoff = -0.01:0.01 deg;\\
mx.yoff = -0.01:0.01 deg;\\
mx.beta = msz.beta;\\
mx.thetaCore = msz.thetaCore;\\
\\
sz.exclude(mx);\\
xray.exclude(msz);
\end{myindentpar}

defines a beta model with independent offsets between the radio and
xray datasets, but with linked \code{beta} and \code{thetaCore}
parameters.

\subsection{Global Control Parameters}
\label{sec:control}

In addition to commands declaring datasets and models,
\climax\ parameters files support global parameters that control what
\climax\ does with the parameter file.  

\subsubsection{Running Markov Chains}

By default, \climax\ will attempt to run a Markov chain for any
variable model components.  The total length of the chain and the
length of the burn-in sequence can be controlled by parameters
\code{ntry} amd \code{nburn}, respectively.

By default, \climax\ will attempt to display parameter histogram
plots, data and residuals with the best-fit model at the end of the
chain.  You can use the \code{dev} parameter to redirect the plots to
device other than the screen (\code{\slash dev\slash null} or a
hardcopy device).  \climax\ uses \code{pgplot} for graphics and
expects the usual pgplot device specifications.  For example, to run a
chain with 10000 total iterations and burn-in length of 3000, and
redirect the plot to a postscript file \code{climax.ps}, do

\begin{myindentpar}{3cm}
ntry = 10000;\\
nburn = 3000;\\
dev = climax.ps/vcps;
\end{myindentpar}

\subsubsection{Displaying Datasets and Models}

If the \code{display} parameter is set to \code{true}, \climax\ will
instead attempt to display any defined datasets, along with residuals
against any defined models. Naturally this requires that models be
defined with fixed parameters, and \climax\ will report an error if
any required model parameters are not defined, or are specified with
priors.

In addition, some datasets support their own \code{display}
parameters, which are executed when the datasets are loaded,
independent of what the parameter file instructs \climax\ to do.  Thus:

\begin{myindentpar}{3cm}
adddataset type=uvf name=duvf;\\
duvf.display = true;
\end{myindentpar}

will cause the dataset to be displayed first, whether running a Markov
chain or creating residual plots.

\subsubsection{Outputting Chains}

By default, parameter chains are stored internally, and are displayed
as histogram plots at the end of a chain.  The chain is not however
written to disk unless you explicitly instruct \climax\ to do so.
Both of these behaviors can be modified by using the \code{store} and
\code{output} keywords.  To turn off internal storage (say, for a very
long chain, for which storage would create a large memory footprint), use

\begin{myindentpar}{3cm}
store = false;
\end{myindentpar}

To output the chain to a file, use:

\begin{myindentpar}{3cm}
output file=/path/to/my/output/filename;
\end{myindentpar}

\subsubsection{Reading Chains Back into \climax}

You can instruct \climaxb\ to load chain files written by \climax\ (or
even \markov) by using the command

\begin{myindentpar}{3cm}
load file=/path/to/my/output/filename;
\end{myindentpar}

This will load the file, in whatever units the parameters are
specified, and create parameter histograms, as at the end of an
internal \climax\ Markov chain run.

\newpage
\subsection{Examples}

\begin{myindentpar}{3cm}
\small
\begin{verbatim}
//============================================================
// An example of fitting a beta model to short-baseline SZA data
//============================================================

//------------------------------------------------------------
// Add the data set
//------------------------------------------------------------

adddataset name=duvf type=uvf;
duvf.file = ~eml/projects/climax/climaxTestSuite/A1914.uvf;
duvf.uvmin = 0;
duvf.uvmax = 2000;

//------------------------------------------------------------
// Add the beta model
//------------------------------------------------------------

addmodel name=m_cluster type=betamodel;

//------------------------------------------------------------
// Assign values to fixed model parameters
//------------------------------------------------------------

m_cluster.beta = 0.8;
m_cluster.axialRatio = 1;
m_cluster.rotang    = 0 degrees;
m_cluster.spectralType = sz;
m_cluster.normalizationFrequency = 30 GHz;

//------------------------------------------------------------
// Assign priors to parameters we want to fit
//------------------------------------------------------------

m_cluster.thetaCore = 5:295";
m_cluster.Sradio = -5:0 mK;
m_cluster.xoff   = -60:60";
m_cluster.yoff   = -60:60";

//------------------------------------------------------------
// Directives controlling the run itself
//------------------------------------------------------------

ntry  = 10000;
nburn = 3000;
nbin  = 50;

//------------------------------------------------------------
// Display to Pgplot window 1
//------------------------------------------------------------

dev = 1/xs;
\end{verbatim}
\end{myindentpar}

